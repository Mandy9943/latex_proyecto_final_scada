% En esta sección, introduciríamos el propósito y los objetivos del proyecto. Deberías describir brevemente el proceso de la planta de tratamiento de agua, y por qué has elegido concentrarte en el subsistema de ósmosis inversa
\chapter*{Introducción}
\addcontentsline{toc}{chapter}{Introducción}

El presente informe detalla el diseño e implementación de un sistema de control y adquisición de datos supervisado (SCADA, por sus siglas en inglés) para una planta de tratamiento de agua, con especial énfasis en el subsistema de ósmosis inversa.\\

El agua es un recurso vital, pero su disponibilidad en forma segura y utilizable no es universal. Para abordar este desafío, se han desarrollado tecnologías avanzadas de tratamiento de agua, entre las cuales, la ósmosis inversa juega un papel crucial. Esta técnica permite la eliminación eficaz de impurezas y contaminantes, produciendo agua de alta calidad que es segura para el consumo humano y otros usos.\\

El tratamiento de agua mediante ósmosis inversa es un proceso complejo y multifacético, que requiere un control preciso y constante para asegurar su eficacia y eficiencia. Para esta tarea, los sistemas SCADA son una herramienta invaluable. Proporcionan una interfaz de usuario para el monitoreo y control del proceso, además de características esenciales como el manejo de eventos y alarmas, registro e historial de datos, y comunicación con el sistema de control.\\

El software TIA Portal de Siemens, que es una plataforma integral para la automatización industrial, fue utilizado para el desarrollo e implementación del sistema SCADA. Este software permite una integración sin fisuras de todos los componentes del sistema, desde los sensores y actuadores en el campo hasta las interfaces de usuario y sistemas de gestión de datos.\\

El objetivo de este proyecto es proporcionar un control preciso y constante del proceso de ósmosis inversa, permitiendo la detección y corrección temprana de problemas, así como la optimización continua del proceso. Este informe documentará cada paso en el diseño e implementación del sistema, con capturas de pantalla y explicaciones detalladas de cada componente y su función.