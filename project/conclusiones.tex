\chapter*{Conclusiones}
\addcontentsline{toc}{chapter}{Introducción}

En este proyecto, hemos implementado un sistema SCADA para controlar y monitorear un subsistema de ósmosis inversa en una planta de tratamiento de agua utilizando el software TIA Portal y un PLC Siemens S7-1200. El sistema ha sido diseñado y configurado para cumplir con los requerimientos de sinópticos, manejo de eventos, registro de datos, historial de datos y eventos, y comunicación con el sistema de control.\\

El SCADA que desarrollamos permite una visión global del proceso y brinda una interfaz de usuario intuitiva para la monitorización y control del sistema. Con la implementación de alarmas, podemos detectar y responder rápidamente a condiciones anormales para garantizar la seguridad y la eficiencia del proceso.\\

Hemos registrado y almacenado datos críticos del proceso y eventos en archivos dentro del TIA Portal. Estos datos pueden ser accedidos en el futuro para análisis y diagnósticos, lo que permite una mejora continua del proceso.\\

Además, se ha implementado una administración de usuarios con diferentes niveles de acceso, asegurando que las operaciones críticas del sistema solo sean realizadas por personal autorizado.\\

Por último, la comunicación entre el PLC y el sistema SCADA se logró a través de una red Profinet, lo que permite un control eficiente y una rápida transmisión de datos.\\

Como resultado, este proyecto no solo proporciona una solución de control y supervisión efectiva para una planta de tratamiento de agua, sino que también demuestra la versatilidad y eficiencia de los sistemas SCADA en la gestión de procesos industriales.\\

En el futuro, este trabajo puede ser ampliado para incluir otras partes de la planta de tratamiento de agua, así como la implementación de técnicas de control más avanzadas y sistemas de diagnóstico y predicción basados en inteligencia artificial.